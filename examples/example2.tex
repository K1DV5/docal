\documentclass[11pt]{article}

\usepackage{amsmath}
\usepackage{graphicx}
\begin{document}

\section{Design Analysis}

The full assembly illustration is shown below.

\subsection{Blade Thickness}

Assuming the angle that the "cement bumps" make with the vertical is $30^\circ$,

And from \cite[Figure 9.10]{shigley}, the force that a human can exert with one hand vertically down is between 80 N and 116 N. Since two people will pull the blade down with one hand each, the total verticalforce exerted on the blade will be the addition of these forces.

\begin{equation}
F_{B}	= 196\,\mathrm{N}
\end{equation}

Then the cement will react with the same force distributed evenly along the blade. Since the blade has a length of  , the vertical reaction of the cement will be:

\begin{align}
\begin{split}
R_{c}	&= \frac{F_{B}}{1\,\mathrm{m}}\\
		&= \frac{196\,\mathrm{N}}{1\,\mathrm{m}}\\
		&= 196\,\mathrm{{N}\slash{m}}\\
\end{split}
\end{align}

Since the blade will be supported on both ends, from \cite[Table A-9]{shigley}, it can be assumed as number 7, simple supports, uniform load system. then taking the dimension 
 $b = 10\,\mathrm{cm}$ , the stress equations will be solved for h. The Area moment of inertia for the blade is:

\begin{equation}
i=\frac{b}{12} \, h^{3}
\end{equation}

And the center of mass is:

\begin{equation}
c=\frac{h}{2}
\end{equation}

From \cite[Table A-9]{shigley}, 7, The maximum bending moment equation can be found, then from that equation, h can be solved for.

\begin{equation}
l	= 1\,\mathrm{m}
\end{equation}

\begin{align}
\begin{split}
x	&= \frac{l}{2}\\
	&= \frac{1\,\mathrm{m}}{2}\\
	&= 0.5\,\mathrm{m}\\
\end{split}
\end{align}

\begin{align}
\begin{split}
M_{B}	&= \frac{R_{c}}{2} \cdot x \cdot \left(l - x\right)\\
		&= \frac{196\,\mathrm{{N}\slash{m}}}{2} \times 0.5\,\mathrm{m} \times \left(1\,\mathrm{m} - 0.5\,\mathrm{m}\right)\\
		&= 24.5\,\mathrm{N \, m}\\
\end{split}
\end{align}

The blade material is [Table A21] steel AISI 1040 Q\&T $205^\circ C$ that has a tensile strength

\begin{equation}
S_{B}	= 779\,\mathrm{MPa}
\end{equation}

Taking a factor of safety of:

\begin{equation}
n	= 2
\end{equation}

The Thicakess of the blade can now be solved from:

\begin{equation}
\frac{S_{B}}{n}=- i \, M_{B} \, c
\end{equation}

where

\begin{equation}
i=\frac{b}{12} \, h^{3}
\end{equation}

\begin{equation}
c=\frac{h}{2}
\end{equation}

\begin{align}
\begin{split}
t_{b}	&= \sqrt{6} \cdot \sqrt{\frac{M_{B} \cdot n}{S_{B} \cdot b}}\\
		&= \sqrt{6} \times \sqrt{\frac{24.5\,\mathrm{N \, m} \times 2}{779\,\mathrm{MPa} \times 10\,\mathrm{cm}}}\\
		&= 1.943\,\mathrm{mm}\\
\end{split}
\end{align}

Therefore the selected thickness of the blade is:

\begin{align}
\begin{split}
t_{b}	&= 2.5 \cdot 1\,\mathrm{mm}\\
		&= 2.5 \times 1\,\mathrm{mm}\\
		&= 2.5\,\mathrm{mm}\\
\end{split}
\end{align}

\section{Side Rails}

The side rails are made of aluminum for easy removal of dry cement which can obstruct the movement of the blade wheels. The selected aluminum [Table A24] is wrought, 2024, T3 tempered with a strength of 

\begin{equation}
S_{Al}	= 482\,\mathrm{MPa}
\end{equation}

The profile of the rails can be approximated as

\includegraphics{profile.png}

where 
 $b = 5\,\mathrm{cm}$ and 
 $h = 11\,\mathrm{cm}$ Then the thickness t must be found. The initial guess for the thickness is
 $t = 1\,\mathrm{mm}$ 

The center of mass is:

\begin{align}
\begin{split}
I_{R}	&= \frac{b}{12} \cdot h^{3} - \frac{1}{12} \cdot \left(b - 2 \cdot t\right) \cdot \left(h - 2 \cdot t\right)^{3}\\
		&= \frac{5\,\mathrm{cm}}{12} \times 11\,\mathrm{cm}^{3} - \frac{1}{12} \times \left(5\,\mathrm{cm} - 2 \times 1\,\mathrm{mm}\right) \times \left(11\,\mathrm{cm} - 2 \times 1\,\mathrm{mm}\right)^{3}\\
		&= 5.07(10^{-07})\,\mathrm{m^{4}}\\
\end{split}
\end{align}

\begin{align}
\begin{split}
S	&= 5 \cdot 1\,\mathrm{m} + t_{b}\\
	&= 5 \times 1\,\mathrm{m} + 2.5\,\mathrm{mm}\\
	&= 5.003\,\mathrm{m}\\
\end{split}
\end{align}

\end{document}